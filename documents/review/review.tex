\documentclass{article}

\usepackage[ngerman]{babel}
\usepackage[utf8]{inputenc}
\usepackage[T1]{fontenc}
\usepackage{hyperref}
\usepackage{csquotes}

\usepackage[
    backend=biber,
    style=apa,
    sortlocale=de_DE,
    natbib=true,
    url=false,
    doi=false,
    sortcites=true,
    sorting=nyt,
    isbn=false,
    hyperref=true,
    backref=false,
    giveninits=false,
    eprint=false]{biblatex}
\addbibresource{../references/bibliography.bib}

\title{Review des Papers "Ethik im Umgang mit Daten" von Eduard Gabrielyan}
\author{Pietro D'Ambrosio}
\date{\today}

\begin{document}
\maketitle

\abstract{Dies ist ein Review der Arbeit zum Thema Ethik im Umgang mit Daten von Eduard Gabrielyan.}

\section{Review}

\begin{itemize}
\item Kapitel 1: Einleitung
Positive Aspekte:
Klare Definitionen: Das Kapitel beginnt mit einer klaren und verständlichen Definition von KI und maschinellem Lernen, was für den Lesern einen klaren und einfachen Einstieg in das Thema gibt.
Verständliche Erklärungen: Die Erklärung zu maschinellem Lernen und Deep Learning sind eindeutig und verständlich. Besonders die Analogie, dass Maschinen ähnlich wie Kinder durch Erfahrung lernen, erleichtert das Verständnis da es ein gutes Beispiel ist.
Negative Aspekte:
Quellenangaben: Die Quellenangaben sind teilweise unvollständig oder unklar (z.B. „Frauenhofer.de, n.d.“), was die Nachvollziehbarkeit und Vertrauenswürdigkeit der Informationen negativ beeinträchtigt.
Tipp-Fehler: Es gibt einige Tippfehler, die auffalen (z.B. „traniert“ statt „trainiert“, „BEispiel“ statt „Beispiel“).

\item Kapitel 2: Unternehmen und KI, Daten, Ethik 
   
Positive Aspekte: 
Ethische Verantwortung: Das Kapitel betont die ethische Verantwortung von Unternehmen im Umgang mit KI und Daten, was ein sehr wichtiger Aspekt in der aktuellen Diskussion um KI ist.
Praktische Hinweise: Es werden konkrete Fakten genannt, wie Unternehmen ethisch korrekt mit Daten umgehen können, einschließlich der Informationspflichten und des Datenschutzes.

Negative Aspekte:
Tiefe der Analyse: Die Analyse über ethische Verantwortung und Datenethik bleibt relativ oberflächlich. Eine tiefere Analyse oder konkrete Beispiele könnten die Ausführungen bereichern.
Struktur: Die Struktur des Kapitels könnte verbessert werden. Es gibt einige Wiederholungen und die Übergänge zwischen den Abschnitten sind manchmal nicht klar und «flüssig».

\item Kapitel 3: Fazit
Positive Aspekte:
Zusammenfassung der wichtigsten Punkte: Das Fazit fasst das wichtigste gut zusammen und betont die Bedeutung eines verantwortungsvollen Umgangs mit KI und Daten.
Betonung der Kundenrechte: Die Betonung der Wichtigkeit, die Rechte der Kunden zu schützen, ist ein wichtiger und richtiger Punkt.

Negative Aspekte:
Fehlende Tiefe: Das Fazit könnte detaillierter sein. Es werden viele wichtige Punkte erwähnt, aber eine tiefere Reflexion oder konkrete Handlungsempfehlungen fehlen.
Sprachliche Klarheit: Auch hier gibt es einige sprachliche Ungenauigkeiten, die das Verständnis erschweren.

Gesamteindruck
Die Arbeit hat eine gute Einführung in die Thematik der KI und deren ethischen Aufgaben in Unternehmen. Die Definitionen und Erklärungen zu Beginn sind klar und verständlich, und die Betonung auf ethische Verantwortung ist sehr relevant. Allerdings gibt es Verbesserungsmöglichkeiten in Bezug auf die Tiefe der Analyse und die sprachliche Korrektheit. Eine detailliertere Diskussion der ethischen Herausforderungen und konkretere Beispiele könnten die Arbeit verbessern. Zudem sollten die Quellenangaben präziser und konsistenter sein, um die Glaubwürdigkeit zu erhöhen.

Pietro D’Ambrosio
\end{itemize}





\printbibliography

\end{document}
