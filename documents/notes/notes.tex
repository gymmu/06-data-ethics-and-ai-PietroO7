\documentclass{article}

\usepackage[ngerman]{babel}
\usepackage[utf8]{inputenc}
\usepackage[T1]{fontenc}
\usepackage{hyperref}
\usepackage{csquotes}

\usepackage[
    backend=biber,
    style=apa,
    sortlocale=de_DE,
    natbib=true,
    url=false,
    doi=false,
    sortcites=true,
    sorting=nyt,
    isbn=false,
    hyperref=true,
    backref=false,
    giveninits=false,
    eprint=false]{biblatex}
\addbibresource{../references/bibliography.bib}

\title{Notizen zum Projekt Data Ethics}
\author{D'Ambrosio Pietro}
\date{\today}

\begin{document}
\maketitle

\abstract{
    Dieses Dokument ist eine Sammlung von Notizen zu dem Projekt. Die Struktur innerhalb des
    Projektes ist gleich ausgelegt wie in der Hauptarbeit, somit kann hier einfach geschrieben
    werden, und die Teile die man verwenden möchte, kann man direkt in die Hauptdatei ziehen.
}

\tableofcontents

\section{Einführung in KI}

Künstliche Intelligenz (KI) ist eine Technologie, die bisher menschliche 
Interventionen benötigte Prozesse oder Tätigkeiten automatisiert. 
Sie verbessert die Leistung und Effizienz des Unternehmens und ist in der Lage, 
Erkenntnisse aus Daten zu gewinnen, die sonst niemand gewinnen kann.
KI ist für Firmen, die nach einer Steigerung der Effizienz, 
neuen Umsatzmöglichkeiten und einer verbesserten Kundenbindung streben,
ein gute Investitionsgrund.

\input{section_ai.tex}

\printbibliography

\end{document}
