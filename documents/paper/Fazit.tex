\newpage
\chapter{Fazit}
In diesem Kapitel habe ich meine Fazit über, was sind die Risiken von Bias in KI-Systemen
und wie kann man diese minimieren.

\vspace{5mm}Kognitive Verzerrungen sind Voreingenommenheiten oder Abweichungen von klaren 
Standards im menschlichen Verhalten. Diese Verzerrungen beeinflussen 
auch Künstliche Intelligenz (KI), die auf voreingenommenen Daten basieren
kann und somit beispielsweise zu diskriminierenden Entscheidungen führen kann.
Ursachen von Bias umfassen vorgefasste Meinungen, emotionale 
Entscheidungen und historische Ereignisse. Diese Verzerrungen 
betreffen verschiedene Bereiche wie Geschlecht, Alter und Aussehen.
Die Folgen von Bias sind gravierend, wie das Beispiel der 
Gesundheitsversorgung in den USA zeigt, wo Afroamerikaner*innen benachteiligt wurden.
Um Bias in KI zu reduzieren, sollten verschiedene Datenquellen genutzt, 
transparente Algorithmen entwickelt, Ethikrichtlinien implementiert
und Bias-Training durchgeführt werden. Diese Maßnahmen würden jegliche KI sehr fördern 
und es würden mehr faire und ausgewogene KI-Systeme geben die auch in der Gesellschaft besser genutzt werden
können.