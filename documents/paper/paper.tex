\documentclass{report}

\usepackage[ngerman]{babel}
\usepackage[utf8]{inputenc}
\usepackage[T1]{fontenc}
\usepackage{hyperref}
\usepackage{csquotes}
\usepackage[a4paper]{geometry}
\usepackage{graphicx}

\usepackage[
    backend=biber,
    style=apa,
    sortlocale=de_DE,
    natbib=true,
    url=false,
    doi=false,
    sortcites=true,
    sorting=nyt,
    isbn=false,
    hyperref=true,
    backref=false,
    giveninits=false,
    eprint=false]{biblatex}
\addbibresource{../references/bibliography.bib}


\title{Ethik im Umgang mit Daten}
\author{D'Ambrosio Pietro}
\date{\today}


\begin{document}

\maketitle

\abstract{
In diesem Dokument habe ich mich mit KI, Bias und mit der Frage 
beschäftigt:Was sind die Risiken von Bias in KI-Systemen und wie kann man diese minimieren?}

\tableofcontents
\newpage
\chapter{Einleitung}
Eine Perfekte Einleitung ins Thema wäre eine von ChatGPT generierte 
zusammenfassung über Daten, Ethik und KI zum zeigen wie hilfreich dieses Tool ist.
"Künstliche Intelligenz (KI) verwendet Daten, um anhand von Mustern zu 
erlernen und Entscheidungen zu treffen. Dadurch ist KI in vielen Bereichen 
ein mächtiges Instrument. Die Ethik im Bereich der KI behandelt die moralischen 
Fragen und Schwierigkeiten, die sich aus der Anwendung dieser Technologien ergeben, 
vor allem in Bezug auf Fairness, Transparenz und Rechenschaftspflicht. Die Verbindung 
zwischen KI, Daten und Ethik besteht darin, wie Daten gesammelt, genutzt und verarbeitet 
werden, um die faire und unvoreingenommene Entscheidungsfindung von KI-Systemen zu gewährleisten. 
Im Verlauf des Dokuments wird jedoch erläutert, warum dies unmöglich ist. Um sicherzustellen, 
dass KI-Systeme die Rechte und Würde aller Menschen achten, sind ethische Grundsätze von entscheidender 
Bedeutung."



\input{KI_einführung.tex}
\chapter{Kognitive Verzerrung/Bias}
In diesem Kapitel bearbeite ich meine Hauptfrage: 
Was sind die Risiken von Bias in KI-Systemen und wie kann man diese minimieren?
      
     
\includegraphics[width=0.25\textwidth]{images.jpg}

\section{Was ist Kognitive Verzerrung?}

Zur Bestimmung systematischer fehlerhafter kognitiver Verzerrungen 
ist es zunächst notwendig, rationale Vergleichsstandards anhand 
prüfbarer Regeln zu entwickeln. Diese werden je nach Untersuchungsgegenstand 
anhand von normativen Modellen wie der mathematischen Wahrscheinlichkeitstheorie 
oder der Logik formuliert. Ein Vergleichsstandard kann aber auch ein 
faktisches Geschehen sein, das mit der Erinnerung an dasselbe verglichen 
wird (Gedächtnisillusionen). Systematische, also nicht nur individuelle 
und zufällige, Abweichungen von diesen Standards gelten dann als irrational 
oder falsch. Kognitive Verzerrung bezieht sich auf inhärente Verzerrungen oder 
Voreingenommenheiten, die im menschlichen Verhalten auftreten.\citep{Kognitive_Verzerrung}.
\newpage
\section{Wie hängt das mit KI zusammen?}
Bias ist ein wichtiger Aspekt in Bezug auf Künstliche Intelligenz (KI), 
da es zu Vorurteilen und Diskriminierung führen kann. Wenn KI-Systeme nicht 
richtig trainiert sind, können sie unbewusste Vorurteile vorziehen und 
zu unfairen Entscheidungen führen. Es ist daher wichtig, sich mit dem Thema 
Bias in der KI-Ethik auseinanderzusetzen, um sicherzustellen, dass KI-Systeme 
gerecht und diskriminierungsfrei arbeiten, was aber nicht immer möglich ist, 
aufgrund der Anzahl genutzter Daten die das selbe aussagen. Wenn 90\% der Daten 
sagen das kleine Menschen dumm sind, wird die KI auch das sagen und diese Meinung 
vertreten und weiterüberlegund zusammenhänge finden. 

\section{Was sind Ursachen von Bias?}
Einige Ursachen von Bias sind die Beurteilung des Aussehens von Menschen, 
vorgefasste Meinungen, logische Fehlschlüsse und die Entscheidungsfindung basierend
auf Emotionen. Diese können entstehen von Ereignisse von der Vergangenheit, fehlinformationen
von Social Media oder allgemeine fehlende Fähigkeiten Personen richtig einzuschätzen.
Aufgelistet sind Hier die häufigsten Voreingenommenheiten.
\begin{itemize}
    \item Geschlechtsbezogene Vorurteile
    \item Altersdisskriminierung
    \item Voreingenommenheit auf Grund des Namens
    \item Voreingenommenheit auf Grund der Attraktivität
    \item Voreingenommenheit durch Selbstüberschätzung
    \item Wahrnehmungsfehler
    \item Voreingenommenheit auf Grund des Aussehens
\end{itemize} \citep{Ursachen_Bias}
\section{Was sind die Folgen von Bias?}
Ein BEispiel hierführ wäre das Ereigniss aus 2019 in den USA.
In den Vereinigten Staaten wurde eine KI eingesetzt, um die 
Gesundheitsversorgung so effektiv wie möglich zu gestalten. 
Diese dient dazu, Patienten*innen mit speziellem Pflegebedarf zu 
identifizieren. 
Eine Studie, die im Oktober 2019 veröffentlicht wurde, weist jedoch darauf hin, 
dass Afroamerikaner*innen mit derselben Schwere der Erkrankung weniger 
häufig als Weiße für eine zusätzliche Pflege vorgeschlagen worden sind. 
Eine KI ohne Zweifel hätte in der Tat die doppelte Anzahl von Afroamerikaner*innen mit 
speziellem Pflegebedarf erfassen sollen. Wie ist eine KI, die eigentlich 
„farbenblind“ ist, zu diesen Resultaten gelangt?
Das Risiko der Patient*innen wurde durch den Algorithmus 
dieser KI als die voraussichtlichen Ausgaben für eine weitere Behandlung
festgelegt. Da der US-amerikanische Staat aufgrund verschiedener Faktoren weniger Geld für die 
Behandlung von Afroamerikaner*innen ausgibt, wurden die Ausgaben und damit auch das 
Risiko für Personen dieser ethnischen Gruppe als niedriger bewertet.
\citep{Folgen_Bias}
\section{Was sind Lösungen von Bias?}
\begin{enumerate}
    \item Verschiedene Datenquellen verwenden: Es ist wesentlich, Daten aus
    unterschiedlichen Quellen und von verschiedenen Bevölkerungsgruppen zu 
    sammeln, um zu gewährleisten, dass KI-Modelle nicht verzerrt trainiert sind.
    Ein vielfältiger Datensatz minimiert das Risiko der Diskriminierung bestimmter Gruppen.
    Dies macht die Modelle widerstandsfähiger und allgemeiner. Andererseits auch sollte
    der Datensatz ständig überprüft und aktualisiert werden, um neue Bias-Trends 
    zu verhindern.

    \item Transparente Algorithmen entwickeln: Transparenz in der Entwicklung von Algorithmen 
    bedeutet, dass die Funktionsweise und Entscheidungsprozesse der KI für alle verständlich 
    und nachvollziehbar sind. Dies fördert das Vertrauen der Nutzer in die Technologie. 
    Durch die Offenlegung der Algorithmen können externe Experten die Systeme überprüfen 
    und eventuelle Bias identifizieren und korrigieren. Transparente Systeme erleichtern 
    auch die Einhaltung gesetzlicher Vorschriften und ethischer Standards.

    \item Ethikrichtlinien implementieren: Die Implementierung klarer ethischer Richtlinien für 
    den Einsatz von KI ist entscheidend, um sicherzustellen, dass die Technologie im Einklang 
    mit gesellschaftlichen Werten steht. Diese Richtlinien sollten auf international anerkannten 
    Prinzipien wie Fairness, Transparenz und Verantwortlichkeit basieren. Unternehmen und Entwickler 
    müssen sich verpflichten, diese Richtlinien in ihrer täglichen Arbeit zu befolgen. Regelmäßige 
    Schulungen und Workshops zu ethischen Fragestellungen können dabei helfen, das Bewusstsein für Bias und
    ethische Verantwortung zu schärfen.

    \item Bias-Training und Bewusstsein: Schulungen zur Sensibilisierung für unbewusste Voreingenommenheit
     sind wichtig, um Entwickler und Anwender von KI-Systemen auf mögliche Bias aufmerksam zu machen.
     Diese Schulungen sollten praktische Beispiele und Szenarien enthalten, um das Verständnis zu vertiefen. 
     Ein Bewusstsein für Bias trägt dazu bei, dass sowohl Daten als auch Modelle kritisch überprüft werden.
     Langfristig fördert dies die Entwicklung fairer und ausgewogener KI-Systeme, die die Vielfalt der Gesellschaft 
     widerspiegeln.
\end{enumerate}


\printbibliography

\end{document}
