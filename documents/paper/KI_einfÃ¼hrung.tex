
\newpage
\chapter{Künstliche Intelligenz}
\section{Was ist KI?}
Künstliche Intelligenz (KI) ist eine Technologie, die bisher menschliche 
Interventionen benötigte Prozesse oder Tätigkeiten automatisiert. 
Sie verbessert die Leistung und Effizienz des Unternehmens und ist in der Lage, 
Erkenntnisse aus Daten zu gewinnen, die sonst niemand gewinnen kann.
KI ist für Firmen, die nach einer Steigerung der Effizienz, 
neuen Umsatzmöglichkeiten und einer verbesserten Kundenbindung streben,
ein gute Investitionsgrund.

\section{Wie funktioniert KI}
Die KI-Technologie verbessert die Leistung und Effizienz eines Unternehmens,
indem sie Verfahren oder Aufgaben automatisiert, die bisher menschliches Eingreifen
erforderten. Darüber hinaus ist es der KI möglich, aus Daten Erkenntnisse zu gewinnen,
zu denen kein Mensch fähig wäre.

\section{Ist KI gefährlich?}
Das kommt drauf an wie man es betrachtet. Je nachdem, wie KI genutzt wird,
kann sie sowohl diskriminierend als auch lebensrettend wirken. 
Jedoch bestehen auch Risiken, etwa die Verwendung von Algorithmusbewertungen 
zur Risikobewertung geschlechtsspezifischer Gewalt. Es ist von großer 
Bedeutung, die möglichen Gefahren und Vorzüge von KI sorgfältig zu untersuchen 
und dafür zu sorgen, dass sie verantwortungsvoll genutzt wird.

\section{Wie wird KI trainiert?}
Eine dreistufige Methode zur Schulung von KI-Systemen umfasst das Training,
die Validierung und die Verbesserung. Für das Training von KI existieren zwei
primäre Ansätze: das überwachte Lernen mit gekennzeichneten Daten und das nicht 
überwachte Lernen ohne gekennzeichnete Daten. Daten werden während des Trainings 
in das System eingespeist, um Prognosen zu machen und die Präzision zu beurteilen. 
Bei der Validierung erfolgt eine Bewertung der Performance des trainierten Modells 
anhand neuer Daten. Es wird ständig daran gearbeitet, die Geschwindigkeit und Genauigkeit 
des Modells zu verbessern.


