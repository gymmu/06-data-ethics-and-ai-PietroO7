\section{Was sind die Folgen von Bias?}
Ein BEispiel hierführ wäre das Ereigniss aus 2019 in den USA.
In den Vereinigten Staaten wurde eine KI eingesetzt, um die 
Gesundheitsversorgung so effektiv wie möglich zu gestalten. 
Diese dient dazu, Patienten*innen mit speziellem Pflegebedarf zu 
identifizieren. 
Eine Studie, die im Oktober 2019 veröffentlicht wurde, weist jedoch darauf hin, 
dass Afroamerikaner*innen mit derselben Schwere der Erkrankung weniger 
häufig als Weiße für eine zusätzliche Pflege vorgeschlagen worden sind. 
Eine KI ohne Zweifel hätte in der Tat die doppelte Anzahl von Afroamerikaner*innen mit 
speziellem Pflegebedarf erfassen sollen. Wie ist eine KI, die eigentlich 
„farbenblind“ ist, zu diesen Resultaten gelangt?
Das Risiko der Patient*innen wurde durch den Algorithmus 
dieser KI als die voraussichtlichen Ausgaben für eine weitere Behandlung
festgelegt. Da der US-amerikanische Staat aufgrund verschiedener Faktoren weniger Geld für die 
Behandlung von Afroamerikaner*innen ausgibt, wurden die Ausgaben und damit auch das 
Risiko für Personen dieser ethnischen Gruppe als niedriger bewertet.
\citep{Folgen_Bias}