
\section{Was ist Kognitive Verzerrung?}

Zur Bestimmung systematischer fehlerhafter kognitiver Verzerrungen 
ist es zunächst notwendig, rationale Vergleichsstandards anhand 
prüfbarer Regeln zu entwickeln. Diese werden je nach Untersuchungsgegenstand 
anhand von normativen Modellen wie der mathematischen Wahrscheinlichkeitstheorie 
oder der Logik formuliert. Ein Vergleichsstandard kann aber auch ein 
faktisches Geschehen sein, das mit der Erinnerung an dasselbe verglichen 
wird (Gedächtnisillusionen). Systematische, also nicht nur individuelle 
und zufällige, Abweichungen von diesen Standards gelten dann als irrational 
oder falsch. Kognitive Verzerrung bezieht sich auf inhärente Verzerrungen oder 
Voreingenommenheiten, die im menschlichen Verhalten auftreten.\citep{Kognitive_Verzerrung}.
\newpage
\section{Wie hängt das mit KI zusammen?}
Bias ist ein wichtiger Aspekt in Bezug auf Künstliche Intelligenz (KI), 
da es zu Vorurteilen und Diskriminierung führen kann. Wenn KI-Systeme nicht 
richtig trainiert sind, können sie unbewusste Vorurteile vorziehen und 
zu unfairen Entscheidungen führen. Es ist daher wichtig, sich mit dem Thema 
Bias in der KI-Ethik auseinanderzusetzen, um sicherzustellen, dass KI-Systeme 
gerecht und diskriminierungsfrei arbeiten, was aber nicht immer möglich ist, 
aufgrund der Anzahl genutzter Daten die das selbe aussagen. Wenn 90\% der Daten 
sagen das kleine Menschen dumm sind, wird die KI auch das sagen und diese Meinung 
vertreten und weiterüberlegund zusammenhänge finden. 